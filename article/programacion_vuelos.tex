\documentclass[letterpaper]{article}

\usepackage[spanish]{babel}
\usepackage[T1]{fontenc}
\usepackage{authblk}
\usepackage[margin=1in]{geometry}
\usepackage[utf8]{inputenc}
\usepackage{graphicx}
\usepackage{listings}
\usepackage{xcolor}

\lstset{
    backgroundcolor=\color{white},
    basicstyle=\ttfamily\scriptsize, % Ajustar el tamaño de la fuente
    breaklines=true,
    breakatwhitespace=true,
    commentstyle=\color{gray}\itshape,
    frame=single,
    keywordstyle=\color{blue}\bfseries,
    language=Scala,
    numbers=left,
    numbersep=10pt,
    numberstyle=\tiny\color{gray},
    showspaces=false,
    showstringspaces=false,
    stepnumber=1,
    stringstyle=\color{red},
}

% Carpeta de imágenes
\graphicspath{ {./images/} }

\title{
    Informe de Proyecto de curso \\ 
    \textbf{Problema de planificación de vuelos} \\
    Fundamentos de programación funcional y concurrente \\
    \bigskip\includegraphics[scale=0.07]{logo}
}


\author[1]{Elkin Samir Angulo Panameño}
\author[2]{Miguel Angel Muñoz Piñeros}
\author[3]{Jose Miguel Fuertes Benavidez}
\author[4]{Reider Andrés Muñoz Herrera}

% \affil[ ]{ \includegraphics[scale=0.07]{logo} }
\affil[ ]{Universidad del Valle}
\affil[ ]{Facultad de Ingeniería}
\affil[ ]{Escuela de Ingeniería de Sistemas y Computación}

\date{Junio de 2024} 

\begin{document}

\maketitle

\section{Introducción}

A continuación se presenta un informe sobre el uso de técnicas de programación funcional
y concurrente en el desarrollo de un proyecto de curso.
El proyecto consiste en la implementación de un sistema de planificación de vuelos para una aerolínea.
Para ello se desarrollaron funciones con técinicas de programación funcional, las cuales permiten el agil manejo de los vuelos.

\section{Definición de clases y funciones auxiliares}

Antes de comenzar a explicar las técnicas utilizadas, es importante mencionar unas deficiones de clases y 
funciones auxiliares que se utilizarán en el desarrollo del proyecto.

\begin{lstlisting}[caption={Definición de la clase Aeropuerto y clase Vuelo}, label={lst:claseAeropuertoclaseVuelo}, captionpos=b]
case class Aeropuerto(Cod: String, X: Int, Y: Int, GMT: Int)
case class Vuelo( Aln: String, Num: Int, Org: String, HS: Int, MS: Int, Dst: String, HL: Int, ML: Int, Esc: Int)
\end{lstlisting}

La clase Aeropuerto tiene los atributos Cod, X, Y y GMT, que representan el código del aeropuerto, la posición en el eje X, 
la posición en el eje Y y la diferencia horaria con respecto al GMT, respectivamente. 
Y la clase Vuelo tiene los atributos Aln, Num, Org, HS, MS, Dst, HL, ML y Esc, que representan 
la aerolínea, el número de vuelo, el aeropuerto de origen, la hora de salida, los minutos de salida, 
el aeropuerto de destino, la hora de llegada, los minutos de llegada y el número de escalas, respectivamente.

\begin{lstlisting}[caption={Función para obtener el tiempo en GMT}, label={lst:obtenerGMT}, captionpos=b]
def obtenerGMT(aeropuertos: List[Aeropuerto], aeropuerto: String): Int = {
    (aeropuertos.find(_.Cod == aeropuerto).get.GMT / 100).toInt
}
\end{lstlisting}

La funcion \texttt{obtenerGMT} recibe una lista de aeropuertos y un aeropuerto, y retorna la diferencia horaria con respecto al GMT.

\section{Informe de uso de técnicas de programación funcional y concurrente}

En esta sección se describen las técnicas de programación funcional y concurrente utilizadas en el desarrollo del proyecto.
Y se argumenta por qué se considera que estas técnicas son adecuadas para el problema de planificación de vuelos. 

\subsection{Uso de técnicas de programación funcional}




\begin{lstlisting}[caption={Función para obtener el tiempo de vuelo}, label={lst:obtenerTiempoVuelo}, captionpos=b]
def obtenerTiempoVuelo( aeropuertos: List[Aeropuerto], itinerario: Itinerario ): Int = {    
    itinerario.foldRight(0)((vuelo, acc) => {
      val (vOrgGMT, vDstGMT) = (obtenerGMT(aeropuertos, vuelo.Org), obtenerGMT(aeropuertos, vuelo.Dst))
      val HSvGMT = if (vuelo.HS - vOrgGMT < 0) (vuelo.HS - vOrgGMT + 24) else (vuelo.HS - vOrgGMT)
      val HLvGMT = if (vuelo.HL - vDstGMT < 0) (vuelo.HL - vDstGMT + 24) else (vuelo.HL - vDstGMT)
      val diferenciaHvGMT = ((HLvGMT * 60 + vuelo.ML) - (HSvGMT * 60 + vuelo.MS))
      if ( diferenciaHvGMT < 0 ) acc +  diferenciaHvGMT + (24 * 60) else acc + diferenciaHvGMT
    })
}
\end{lstlisting}

Lorem ipsum dolor sit amet, consectetur adipiscing elit.
Nullam auctor, nunc nec ultricies ultricies, nunc nunc
suscipit nunc, nec ultricies nunc nunc nec.

\subsection{Uso de técnicas de paralelización de tareas y datos}

Lorem ipsum dolor sit amet, consectetur adipiscing elit.
Nullam auctor, nunc nec ultricies ultricies, nunc nunc
suscipit nunc, nec ultricies nunc nunc nec.




\end{document}